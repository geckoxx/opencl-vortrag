\documentclass{beamer}
\usepackage[utf8x]{inputenc}
\usepackage[ngerman]{babel}
\usepackage{amsmath}
\usepackage{amsfonts}
\usepackage{amssymb}
\usepackage{graphicx}
\usepackage{hyperref}
\usepackage{listings}
\lstset{literate=%
{Ö}{{\"O}}1
{Ä}{{\"A}}1
{Ü}{{\"U}}1
{ß}{{\ss}}2
{ü}{{\"u}}1
{ä}{{\"a}}1
{ö}{{\"o}}1
}

\lstset{language=C}

\author{Johannes Hackel \and Tim Illner \and Marcel Wesberg}
\title{OpenCL}

\usetheme{Ilmenau}
\useoutertheme{infolines}
\usecolortheme{rose}

\begin{document}

\begin{frame}
\titlepage
\end{frame}

\begin{frame}
\frametitle{Gliederung}
\tableofcontents
\end{frame}

\section{Was ist OpenCL?}
\begin{frame}[fragile]
\frametitle{Was ist OpenCL?}
\url{http://en.wikipedia.org/wiki/OpenCL}
\end{frame}
\subsection{Geschichte}
\begin{frame}[fragile]
\frametitle{Geschichte}
\url{http://en.wikipedia.org/wiki/OpenCL#History}
\end{frame}
\subsection{Unterstütze Geräte/Treiber}
\begin{frame}[fragile]
\frametitle{Unterstütze Geräte/Treiber}
\url{http://en.wikipedia.org/wiki/OpenCL#OpenCL-conformant_products}
\end{frame}

\section{Aufbau GPUs}
\begin{frame}[fragile]
\frametitle{Aufbau GPUs}
\url{http://www.zdnet.de/wp-content/uploads/legacy_images/news/201004/aws-gpu-v6.png}\\
oder ähnliches
\end{frame}

\section{Die OpenCL}
\subsection{Kernel}
\begin{frame}[fragile]
\frametitle{Die OpenCL}
\framesubtitle{Kernel}
\begin{lstlisting}
float Sum(float x, float y)
{
 return(x + y);
}
__kernel void Calculate(__global float* input,
 __global float* output)
{
 \\Code-Formatierungstest
}
\end{lstlisting}
\end{frame}

\subsection{Datentypen}
\begin{frame}[fragile]
\frametitle{Datentypen}
\end{frame}
\subsection{Speicherbereiche}
\begin{frame}[fragile]
\frametitle{Speicherbereiche}
\end{frame}
\subsection{Funktionen}
\begin{frame}[fragile]
\frametitle{Funktionen}
\end{frame}

\section{Die Laufzeitbibliothek}
\begin{frame}[fragile]
\frametitle{Die Laufzeitbibliothek}
\end{frame}
\subsection{Plattformen/Geräte}
\begin{frame}[fragile]
\frametitle{Plattformen/Geräte}
\end{frame}
\subsection{Kontexte/Speicherverwaltung}
\begin{frame}[fragile]
\frametitle{Kontexte/Speicherverwaltung}
\end{frame}
\subsection{Programme/Kernels}
\begin{frame}[fragile]
\frametitle{Programme/Kernels}
\end{frame}
\subsection{Warteschlangen/Ereignisse/Marker und Barrieren}
\begin{frame}[fragile]
\frametitle{Warteschlangen/Ereignisse/Marker und Barrieren}
\end{frame}

\section{Kompilierung und Ausführung}
\begin{frame}[fragile]
\frametitle{Kompilierung und Ausführung}
\end{frame}
\end{document}