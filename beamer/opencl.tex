\documentclass{beamer}
\usepackage[latin1]{inputenc}
\usepackage[ngerman]{babel}
\usepackage{amsmath}
\usepackage{amsfonts}
\usepackage{amssymb}
\usepackage{graphicx}
\usepackage{listings}
\lstloadlanguages{C}
\author{Johannes Hackel}
\title{OpenCL}

\usetheme{Ilmenau}
\useoutertheme{infolines}
\usecolortheme{rose}

\begin{document}

\begin{frame}
\titlepage
\end{frame}

\begin{frame}
\frametitle{Gliederung}
\tableofcontents
\end{frame}

\section{Was ist OpenCL?}
\subsection{Geschichte}
\subsection{Unterstütze Geräte/Treiber}

\section{Aufbau GPUs}

\section{Die OpenCL}
\subsection{Kernel}
\begin{frame}
\begin{lstlisting}
float Sum(float x, float y)
{
 return(x + y);
}
__kernel void Calculate(__global float* input, __global float* output)
{
 // TODO: Führe Berechnung aus und benutze Funktion Sum.
}

\end{lstlisting}

\end{frame}
\subsection{Datentypen}
\subsection{Speicherbereiche}
\subsection{Funktionen}

\section{Die Laufzeitbibliothek}
\subsection{Plattformen/Geräte}
\subsection{Kontexte/Speicherverwaltung}
\subsection{Programme/Kernels}
\subsection{Warteschlangen/Ereignisse/Marker und Barrieren}

\section{Kompilierung und Ausführung}
\end{document}