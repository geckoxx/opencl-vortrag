\documentclass{beamer}
\usepackage[utf8x]{inputenc}
\usepackage[ngerman]{babel}
\usepackage{amsmath}
\usepackage{amsfonts}
\usepackage{amssymb}
\usepackage{graphicx}
\usepackage{subfigure}
\author{Johannes Hackel und Falco Prescher}
\title{Bildverarbeitung mit OpenCL}

\usetheme{Ilmenau}
\useoutertheme{split}
\usecolortheme{rose}

\newcommand*\oldmacro{}%
\let\oldmacro\insertshorttitle%
\renewcommand*\insertshorttitle{%
  \oldmacro\hfill%
  \insertframenumber\,/\,\inserttotalframenumber}

\begin{document}

\begin{frame}
\titlepage
\end{frame}

\begin{frame}
\frametitle{Gliederung}
\tableofcontents
\end{frame}

\section{OpenCL}

\subsection{Allgemeines zu OpenCL}
\begin{frame}
\frametitle{Allgemeines zu OpenCL}
\begin{itemize}
\item Test
\end{itemize}
\end{frame}

\subsection{Grundlegender Aufbau eines OpenCL-Programmes}
\begin{frame}
\frametitle{Grundlegender Aufbau eines OpenCL-Programmes}
\begin{itemize}
\item Test
\end{itemize}
\end{frame}

\subsection{Vergleich von OpenCL mit CUDA}
\begin{frame}
\frametitle{Vergleich von OpenCL mit CUDA}
\begin{itemize}
\item Test
\end{itemize}
\end{frame}

\section{Bildverarbeitung}

\subsection{Allgemeines zu Bildverarbeitung mit OpenCL}
\begin{frame}
\frametitle{Allgemeines zu Bildverarbeitung mit OpenCL}
\begin{itemize}
\item Test
\end{itemize}
\end{frame}

\subsection{Kantenerkennung von Bildern in OpenCL}
\begin{frame}
\frametitle{Kantenerkennung von Bildern in OpenCL}
\begin{itemize}
\item Test
\end{itemize}
\end{frame}

\begin{appendix}
\begin{frame}
\frametitle{Quellen}
\begin{itemize}
\item \url{http://www.testtesttest.org/}
\end{itemize}
\end{frame}
\end{appendix}

\end{document}